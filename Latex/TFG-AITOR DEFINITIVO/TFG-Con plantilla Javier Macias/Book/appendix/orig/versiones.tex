%%%%%%%%%%%%%%%%%%%%%%%%%%%%%%%%%%%%%%%%%%%%%%%%%%%%%%%%%%%%%%%%%%%%%%%%%%%
%
% Generic template for TFC/TFM/TFG/Tesis
%
% $Id: versiones.tex,v 1.6 2020/03/24 17:18:13 macias Exp $
%
% By:
%  + Javier Macías-Guarasa. 
%    Departamento de Electrónica
%    Universidad de Alcalá
%  + Roberto Barra-Chicote. 
%    Departamento de Ingeniería Electrónica
%    Universidad Politécnica de Madrid   
% 
% Based on original sources by Roberto Barra, Manuel Ocaña, Jesús Nuevo,
% Pedro Revenga, Fernando Herránz and Noelia Hernández. Thanks a lot to
% all of them, and to the many anonymous contributors found (thanks to
% google) that provided help in setting all this up.
%
% See also the additionalContributors.txt file to check the name of
% additional contributors to this work.
%
% If you think you can add pieces of relevant/useful examples,
% improvements, please contact us at (macias@depeca.uah.es)
%
% You can freely use this template and please contribute with
% comments or suggestions!!!
%
%%%%%%%%%%%%%%%%%%%%%%%%%%%%%%%%%%%%%%%%%%%%%%%%%%%%%%%%%%%%%%%%%%%%%%%%%%%

\chapter{Versiones}
\label{cha:versiones}

En este apartado incluyo el historial de cambios más relevantes de la
plantilla a lo largo del tiempo.

No empecé este apéndice hasta principios de 2015, con lo que se ha
perdido parte de la información de los cambios importantes que ha ido
sufriendo esta plantilla.


\begin{itemize}

  
\item Julio 2021
  \begin{itemize}
    
  \item El salto gigante desde el 2015 no es porque no haya ido
    haciendo cambios, pero no he tenido el tiempo necesario para
    documentarlos.
    
  \item Ahora la plantilla está accesible de dos formas:
    \begin{itemize}
    
    \item En github, por si a los que lo queráis usar os es más fácil
      clonar o hacer un fork. Está disponible en
      \url{https://github.com/JaviMaciasG/PhD-TFM-TFG-LatexTemplate} y
      podéis clonarlo desde
      \url{https://github.com/JaviMaciasG/PhD-TFM-TFG-LatexTemplate.git}. Ojo
      que tiene morralla variada que puede que no os interese.
    \item En mi dropbox, en formatos zip y tgz, accesible en
      \url{https://www.dropbox.com/sh/mm6fwh3ruuuyjz2/AABDUmo7Xj1S968FeJgbmFPva?dl=0}
      y sin la morralla que os decía.
      
    \end{itemize}
    
  \item Reestructuración completa de la estructura de directorios
  \item Soporte para el manejo adecuado de los ``géneros'', para lo
    que hay que definirlos en el fichero de configuración. Creo que es
    completo, pero si veis algún error, dadme un toque.
  \item Soporte completo (por fin) de utf-8, salvo en los ficheros
    .bib que no lo he conseguido.    
  \end{itemize}
  
\item Mayo 2015:
  \begin{itemize}
  \item Hay disponible un \texttt{make bare} para que deje los capítulos
    mondos y lirondos y se pueda escribir desde casi cero sin tener que
    andar borrando manualmente.
  \end{itemize}


\item Abril 2015:
  \begin{itemize}
  \item Ahora manejamos masculino/femenino en algunos sitios (el/la,
    autor/autora, alumno/alumna, del/de la, ...). Hay que definir
    variable con el género del autor (todavía queda pendiente lo de los
    tutores y tal). NOT FINISHED!!
  \end{itemize}


\item Enero 2015:
  \begin{itemize}
  \item Solucionado el problema (gordo) de compilación del
    \texttt{anteproyecto.tex} y el \texttt{book.tex}, debido al uso de
    paths distintos en la compilación de la bibliografía. El sistema se ha
    complicado un poco (ver
    \texttt{biblio\textbackslash{}bibliography.tex}).
  \item Añadido un (rudimentario) sistema para generar pdf con las
    diferencias entre el documento en su estado actual y lo último
    disponible en el repositorio (usando \texttt{latexdiff}).
  \end{itemize}
\item Diciembre 2015:
  \begin{itemize}
  \item Separada la compilación del anteproyecto de la del documento
    principal. Para el primero se ha creado el directorio
    \texttt{anteproyecto} donde está todo lo necesario.
  \end{itemize}
\end{itemize}

%%% Local Variables:
%%% TeX-master: "../book"ve
%%% End:


