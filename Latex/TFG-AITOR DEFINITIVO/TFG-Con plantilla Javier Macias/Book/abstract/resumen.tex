
\chapter*{Resumen}
\label{cha:resumen}
\markboth{Resumen}{Resumen}

\addcontentsline{toc}{chapter}{Resumen}

Desde que Hertz demostrara en 1887 la posibilidad de usar dispositivos para la transmisión de ondas electromagnéticas por un medio abierto como el aire, surgió la necesidad de poder caracterizar su comportamiento de manera fidedigna.

Con el paso del tiempo, los estándares de medida han ido evolucionando para aumentar cada vez más la precisión de la toma de medidas.  En el centro de la motivación de este trabajo está la creación de un algoritmo capaz de representar diagramas de radiación en campo lejano a partir de medidas pertenecientes al campo cercano que permita al centro de Alta Tecnología y Homologación (CATECHOM) de La Escuela Politécnica de la Universidad de Alcalá renovar el software actual que ha estado en uso desde 1999.

\textbf{Palabras clave:} \myThesisKeywords.
