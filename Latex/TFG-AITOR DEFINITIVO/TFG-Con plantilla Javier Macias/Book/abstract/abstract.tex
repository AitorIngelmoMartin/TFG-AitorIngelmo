%%%%%%%%%%%%%%%%%%%%%%%%%%%%%%%%%%%%%%%%%%%%%%%%%%%%%%%%%%%%%%%%%%%%%%%%%%%
%
% Generic template for TFC/TFM/TFG/Tesis
%
% $Id: abstract.tex,v 1.9 2015/06/05 00:10:31 macias Exp $
%
% By:
%  + Javier Macías-Guarasa. 
%    Departamento de Electrónica
%    Universidad de Alcalá
%  + Roberto Barra-Chicote. 
%    Departamento de Ingeniería Electrónica
%    Universidad Politécnica de Madrid   
% 
% Based on original sources by Roberto Barra, Manuel Ocaña, Jesús Nuevo,
% Pedro Revenga, Fernando Herránz and Noelia Hernández. Thanks a lot to
% all of them, and to the many anonymous contributors found (thanks to
% google) that provided help in setting all this up.
%
% See also the additionalContributors.txt file to check the name of
% additional contributors to this work.
%
% If you think you can add pieces of relevant/useful examples,
% improvements, please contact us at (macias@depeca.uah.es)
%
% You can freely use this template and please contribute with
% comments or suggestions!!!
%
%%%%%%%%%%%%%%%%%%%%%%%%%%%%%%%%%%%%%%%%%%%%%%%%%%%%%%%%%%%%%%%%%%%%%%%%%%%

\chapter*{Abstract}
\label{cha:abstract}

\addcontentsline{toc}{chapter}{Abstract}

Since Hertz demonstrated in 1887 the possibility of using devices for the transmission of electromagnetic waves through an open medium such as air, the need arose to be able to characterize their behavior reliably.

Over time, measurement standards have evolved to increase the accuracy of measurements.  At the core of the motivation of this work is the creation of an algorithm capable of representing far-field radiation diagrams from near-field measurements that will allow the High Technology and Homologation Center (CATECHOM) of the Polytechnic School of the University of Alcalá to renew the current software available since 1999.


\textbf{Keywords:} \myThesisKeywordsEnglish.

%%% Local Variables:
%%% TeX-master: "../book"
%%% End:


