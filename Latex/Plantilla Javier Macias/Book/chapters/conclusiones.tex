%%%%%%%%%%%%%%%%%%%%%%%%%%%%%%%%%%%%%%%%%%%%%%%%%%%%%%%%%%%%%%%%%%%%%%%%%%%
%
% Generic template for TFC/TFM/TFG/Tesis
%
% By:
%  + Javier Macías-Guarasa.
%    Departamento de Electrónica
%    Universidad de Alcalá
%  + Roberto Barra-Chicote.
%    Departamento de Ingeniería Electrónica
%    Universidad Politécnica de Madrid
%
% By: + Javier Macías-Guarasa. Departamento de Electrónica Universidad de Alcalá + Roberto Barra-Chicote. Departamento de Ingeniería Electrónica Universidad Politécnica de Madrid
% 
% Based on original sources by Roberto Barra, Manuel Ocaña, Jesús Nuevo, Pedro Revenga, Fernando Herránz and Noelia Hernández. Thanks a lot to all of them, and to the many anonymous contributors found (thanks to google) that provided help in setting all this up.
%
% See also the additionalContributors.txt file to check the name of additional contributors to this work.
%
% If you think you can add pieces of relevant/useful examples, improvements, please contact us at (macias@depeca.uah.es)
%
% You can freely use this template and please contribute with comments or suggestions!!!
%
%%%%%%%%%%%%%%%%%%%%%%%%%%%%%%%%%%%%%%%%%%%%%%%%%%%%%%%%%%%%%%%%%%%%%%%%%%%

\chapter{Conclusiones y líneas futuras}
\label{cha:concl-y-line}

En este apartado se resumen las conclusiones obtenidas y se proponen futuras líneas de investigación que se deriven del trabajo.

La estructura del capítulo es...


\section{Conclusiones}
\label{sec:conclusiones}

Para añadir una referencia a un autor, se puede utilizar el paquete \texttt{cite}. En el trabajo \cite{armani03}, se muestra un trabajo...

Y podemos usar de nuevo algún acrónimo, como por ejemplo \ac{TDPSOLA}, o uno ya referenciado como \ac{ANN} que cambia cuando lo usas una segunda vez \ac{ANN}.

Veamos finalmente si funciona una cita inline... al paper de Leticia \fullcite{monasterio2022}... (esto substituye la funcionalidad de bibentry...).


\section{Líneas futuras}
\label{sec:lineas-futuras}

Pues eso.


%%% Local Variables:
%%% TeX-master: "../book"
%%% End:


